\section{Построение решения задачи}
   
    Готовых средств, решающих задачу, поставленную в данной дипломной работе, нет, поэтому оправданно создание собственного средства.

    За основу разрабатываемой нотации на основе UML было решено взять нотацию, предложенную в статье \cite{mal-manz}, и, если необходимо, дополнить её новыми элементами UML или предложить новый способ представления элементов PDDL.

    С поддержкой преобразования такого подмножества PDDL, как STRIPS, проблем возникнуть не должно.
 Вопрос поддержки возможностей PDDL за пределами STRIPS остается открытым в работе на данный момент.
 Существующие стандарты нуждаются в более детальном изучении.

    Существует средство PDDL4J, в котором заявлено, что реализуется парсер PDDL-описаний и предоставляется Java API для дальнейшего анализа.
 К сожалению, пока непонятно, какую версию PDDL поддерживает это средство.
 Это направлению нуждается в более детальной проработке.
     
    В случае, если PDDL4J нас не устроит (вопросы лицензии, обнаруженные ошибки и т.д.), то вести разработку предполагается опираясь на универсальные фреймворки для создания средств компиляции и анализа.
 Для создания парсера PDDL грамматики возможно использование ANTLR~\cite{antlr}.
 
 \newpage