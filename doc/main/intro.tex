\chapter*{Введение}
\addcontentsline{toc}{chapter}{Введение}

    Одна из задач в области искусственного интеллекта~--- задача планирования.
 Планированием называется процесс выработки последовательности действий, позволяющей достичь цели \cite{norwig-ai}.
 Задача планирования часто встает у агентов, которым нужно найти последовательность действий, выполнив которые он достигнет некоторой своей цели.
 Агентом можно назвать все, что воспринимает свою среду с помощью некоторых датчиков и воздействует на нее с помощью некоторых механизмов.
 Примерами использования планирования может служить следующее: планирование управлением механическими приводами робота-мусоросборника, планирование распределения транспортных средств для перевозок различных видов топлива и сырья на нефтеперерабатывающем заводе.
 
    Контекст задачи планирования, модель мира, в котором возникает задача,  называется предметной областью.
    
    В 1971 году для представления знаний о предметных областях был разработан формальный текстовый язык STRIPS \cite{strips} (STanford Research Institute Problem Solver).
 В этом языке выделяются три основных понятия~--- \textit{состояние среды} (мира), \textit{действие} (механизмы воздействия на среду), \textit{цель} (целевое состояние мира).
 Состояния задаются в виде некоторого набора фактов об объектах и отношениях между ними, которые считаются истинными в этом состоянии.
 Действия задаются с использованием набора ограничений на состояние (предусловие), и набора положительных и отрицательных фактов (эффекты).
 \textit{Предусловие} должно выполняться для того, чтобы применение действия было допустимым в данном состоянии.
 \textit{Эффект} задает то, как меняется состояние при применении действия~--- положительные факты добавляются к состоянию, отрицательные~--- удаляются.
 \textit{Цель}, или целевое состояние, задается как набор ограничений, которые должны быть удовлетворены в данном состоянии.
 Также в STRIPS вводится гипотеза замкнутости мира (CWA, \textit{Closed World Assumption}), которая означает, что факты, не перечисленные в описании состояния, считаются ложными.
    
    В 1987 году был предложен язык ADL (Action Description Language, язык описания действий) похожий на STRIPS, но включающий в себя еще несколько особенностей, из которых можно отметить следующие:
    \begin{compactlist}
        \item предположение об открытом мире - факты, не перечисленные в состоянии, считаются неизвестными;
        \item возможность использования кванторов и дизъюнктов для задания цели;
        \item возможность использования условных эффектов;
        \item наличие встроенного предиката равенства для сравнения объектов;
        \item возможность типизации переменных;
        \item и т.д.
    \end{compactlist}
    
    В 1988 году появился язык PDDL\cite{pddl3} (Planning Domain Definition Language, язык описания предметных областей и задач планирования) как попытка стандартизации существующих на тот момент языков описания предметных областей и задач планирования.
 Также это сделало возможным создание IPC~--- International Planing Competition~--- международных соревнований по созданию планировщиков.
 На данный момент PDDL стал стандартом де-факто для описания знаний систем планирования.
  
    В настоящее время большую популярность приобретают графические представления, основанные на объектных спецификациях.
 В нашем случае, графическое представление знаний может быть более удобным для использования людьми, нежели текстовое представление.
 Для графического моделирования и удовлетворения нужд программной индустрии был разработан язык UML\cite{rambo-uml2} (Unified Modeling Language, унифицированный язык моделирования), но его возможности оказались шире, поэтому он используется и в других областях.
 Например, его можно использовать для представления знаний систем планирования в виде UML-моделей.
 Для задания дополнительных ограничений на UML-модели может быть использован язык OCL\cite{ocl}.
 Для создания UML-моделей существует множество редакторов, что делает возможность задания знаний среды и задачи планирования в графическом представлении более привлекательной.
     
    Применение объектных спецификаций для задания знаний систем планирования является актуальной темой для исследований.
  По этой теме на международной конференции ICAPS (International Conference on Automated Planning and Scheduling) было представлено несколько докладов \cite{icaps-1, icaps-2}.

\newpage